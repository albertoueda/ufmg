\documentclass[a4paper,12pt,titlepage]{article}
%\usepackage[T1]{fontenc}

\usepackage{titlesec}
\usepackage{titling}
\usepackage[portuguese]{babel}
\usepackage[utf8x]{inputenc}
\usepackage{indentfirst}
\usepackage{graphicx}
%\usepackage{times}
\usepackage{ucs}
\usepackage{float}    
\usepackage{fancyvrb}   
\usepackage{verbatim}
\usepackage{listings}
\usepackage{hyperref}
\usepackage{epigraph}
\usepackage{listings}
\usepackage{tabularx}
\usepackage{lipsum}

\hypersetup{
    colorlinks=true,       
    linkcolor=black,          
    citecolor=black,   
    filecolor=black,  
    urlcolor=black  
}

\hyphenation {di-re-cio-na-men-to} 

%\renewcommand*{\familydefault}{\ttdefault}
\lstset{columns=fullflexible,basicstyle=\ttfamily}

\title{\large
Universidade Federal de Minas Gerais \\ \
Instituto de Ciências Exatas \\ \ 
Departamento de Ciência da Computação \\ \
\\[1cm]
Plano de Curso de Doutorado\\ \
\\[1cm]
\textbf{\Large Ordenação de Veículos de Publicação Científica na Área de Humanidades Baseada em Reputação }
\\[1cm]
}

\author{\large Candidato: Alberto Hideki Ueda \\[0.5cm] 
	Orientador: Berthier Ribeiro Neto 
\\[3cm] }

\date{\textsc{Belo Horizonte\\ \
30 de junho de 2014}}

\begin{document}
\maketitle

\renewcommand*\contentsname{Sumário}
\tableofcontents 
\pagebreak

\section{Introdução} 
Com o aumento de produções científicas no mundo e a maior facilidade de divulgação dos mesmos \cite{theguardian}, o número de conferências, jornais e revistas de diversas áreas do conhecimento também se elevou. De tempos em tempos, novos veículos de publicação surgem. Adicionalmente, os veículos existentes ou melhoram sua reputação devido à boa qualidade de trabalhos publicados ou perdem reputação em razão da queda de impacto científico, sendo que alguns chegam a ser descontinuados. Assim, é imperativo ordenar os veículos segundo uma relação de importância, em inglês, \textit{ranking}. %Duvida cruel: torna-se cada vez mais necessária a identificação das conferências e jornais que possuem maior relevância para cada área de pesquisa, ou seja,

Esta identificação deve ser confiável o suficiente para permitir o direcionamento de altos investimentos financeiros para grupos que publicam em veículos de notoriedade, tanto públicos quanto privados, assim como servir de referência para publicações de futuros trabalhos científicos de alto nível. 

Para isso, a maioria dos métodos utilizados para realizar este \textit{ranking} de veículos baseia-se no número de citações aos trabalhos publicados em tais veículos \cite{rahm, mann}. Dentre eles, o \textit{Impact Factor} (IF) \cite{if} se destaca por ser o mais aceito entre os pesquisadores. Embora apresentem resultados satisfatórios, o IF e demais modelos baseados em citações possuem um problema: para a contagem de citações de um determinado trabalho de uma conferência, é necessário que sejam encontrados e acessados todos os documentos científicos que citam tal trabalho. Isto requer repositórios com um grande volume de dados, como o \textit{Google Scholar} ou \textit{ISI Web of Science}, com alto custo de manutenção e ainda suscetíveis a erros de inconsistência \cite{erros1, erros2, erros3, erros4}. Além disso, na prática, nem todas publicações científicas são armazenadas, o que causa um problema de cobertura de dados.

No artigo \textit{``Reputation-based Scoring of Publication Venues''} de Sabir Ribas, doutorando da UFMG, o indicador \textit{R-Score} foi apresentado e implementado para o \textit{ranking} de veículos de publicação nas áreas de Ciência da Computação, Economia e Bioquímica \cite{sabir}. Tal método baseia-se na hipótese de que os grupos de pesquisa de maior reputação em determinada área do conhecimento publicam seus trabalhos nas conferências que também possuem maior reputação. Tais grupos de pesquisa são selecionados inicialmente e chamados de grupos de referência. 

Ao contrário do IF, com o \textit{R-Score} não é encontrar necessário e acessar o conteúdo de todos os trabalhos científicos de determinada área de conhecimento. Isso ocorre pois este indicador não se baseia em citações de outros trabalhos. O que é utilizado como entrada para a ordenação dos veículos são apenas os grupos de referência para a área de conhecimento em questão e as conferências e jornais em que seus trabalhos foram publicados.

Os resultados experimentais discutidos em \cite{sabir}, em comparação com o IF, mostram que: i) não é necessário mais do que dez grupos de pesquisa de referência para que o \textit{ranking} de conferências se estabilize; ii) muitas conferências relevantes que não aparecem nas posições mais altas da ordenação gerada pelo IF ocupam as primeiras posições utilizando-se o indicador \textit{R-Score}; e iii) muitas conferências e jornais interdisciplinares, que claramente não representam relevância à determinada área de pesquisa, estão nas posições mais altas segundo o IF, o que não acontece utilizando-se o \textit{R-Score}.

A idéia inicial deste doutorado é estender os experimentos do trabalho de Sabir Ribas para áreas de humanidades, como Direito e Filosofia. Estas áreas apresentam novos desafios, tais como a alta prevalência de publicações em livros, o que irá requerer novas formulações para o \textit{R-Score}.

\section{Trabalhos Relacionados} 
Atualmente, a abordagem mais utilizada para a ordenação de veículos de publicação científica é o \textit{Impact Factor} de Eugene Garfield \cite{if}, baseado exclusivamente no número de citações. Métricas alternativas que optam pela utilização de mais informações além da contagem de citações às conferências já foram desenvolvidas, como o  \textit{H-Index} \cite{h-index}, algumas abordagens semelhantes ao  \textit{PageRank} \cite{pagerank} e modelos baseados em número de \textit{downloads} do material publicado \cite{downloads}. Para arquivos em bibliotecas digitais, destaca-se o trabalho de Larsen e Ingwersen \cite{larsen}, enquanto para a ordenação de pequenos conjuntos de conferências e jornais encontramos o artigo de Rahm e Thor \cite{rahm}, ambos baseados em número de citações.

Sendo o \textit{ranking} de conferências e jornais científicos uma tarefa importante da bibliometria \cite{biblio}, trabalhos como \cite{mann}, \cite{yan} e \cite{biblio} visam complementar o IF ou mesmo sugerir outros indicadores que apresentam resultados semelhantes.

O indicador \textit{R-Score} proposto por Sabir Ribas avalia as conferências e jornais em termos de reputação. Para mensurar esta reputação, o trabalho pode ser descrito pelas seguintes etapas: a) para a área de conhecimento escolhida para análise, são definidos os grupos de pesquisa de maior reputação no mundo de acordo com avaliações de instituições confiáveis - como o  \textit{National Research Concil} \cite{nrc}, por exemplo; b) lista-se então todas as publicações encontradas destes grupos de referência; c) em seguida, tal lista de publicações é modelada como múltiplas cadeias de Markov; d) calcula-se as probabilidades estacionárias para os estados definidos as quais definem as pontuações de reputação para cada conferência avaliada \cite{sabir}.

Deste modo, o \textit{R-Score} baseia-se diretamente na reputação e influência de pesquisadores e professores dos principais grupos de estudos do mundo em cada área. Pode-se dizer que a reputação destes grupos de referência é transferida às conferências e jornais em que publicam. A principal vantagem do \textit{R-Score} é que ele depende apenas da lista de publicações destes grupos de referência, sem a necessidade de acessar o conteúdo de um grande volume de trabalhos como em indicadores baseados em citações. Esta característica torna o \textit{R-Score} muito mais simples de ser executado em situações reais que o \textit{Impact Factor}, por exemplo. Além disso, muitas conferências ainda não possuem o IF calculado, o que indica um problema de cobertura que o \textit{R-Score} pode solucionar com pouco esforço.

As áreas de conhecimento analisadas utilizando o \textit{R-Score} foram Ciência da Computação, Economia e Bioquímica \cite{sabir}. Embora tenham sido escolhidas por apresentarem padrões de publicação mais simples de se identificar, a ideia do \textit{R-Score} é que ele pode ser aplicado em qualquer área de pesquisa. Para isso, é necessário que sejam identificados os grupos de estudo de maior reputação na área em questão e suas publicações organizadas em um padrão de dados que possa ser utilizado. 

Alguns problemas surgem ao partirmos para análise de áreas de pesquisa em Ciências Humanas. Por exemplo, é comum na área de Direito encontrarmos forte dependência entre a reputação dos trabalhos e a própria história, constituição e cultura do país de origem das publicações \cite{sabir}. Assim como em Sociologia e Filosofia, tais áreas de conhecimento dificilmente possuem conferências prestigiadas de forma unânime pelo mundo, o que faz com que as principais conferências e jornais da área variem consideravelmente de país para país, inclusive em termos de idioma.


\section{Objetivos do Projeto} 
A proposta deste doutorado é estender o trabalho de Sabir Ribas, com os seguintes objetivos:

\begin{itemize}
\item Estudar as dificuldades encontradas para a aplicação do \textit{R-Score} em áreas de estudo de Ciências Humanas, como Direito, Filosofia e Sociologia;
\item Determinar um método formal para a escolha dos grupos de maior reputação nestas áreas, levando-se em consideração fatores como a cultura e o idioma de origem dos trabalhos publicados;
\item Para cada área de conhecimento, organizar em um mesmo padrão de dados as publicações feitas pelos grupos de referências definidas na etapa anterior;
\item Aplicar o indicador \textit{R-Score} nestas áreas e analisar os resultados obtidos, comparando-os com outros \textit{rankings} existentes e opiniões de pesquisadores de cada área de estudo. Poderemos então validar tanto a flexibilidade quanto a abrangência deste indicador para áreas de humanidades.
\end{itemize}

É possível que neste processo o próprio \textit{R-Score} seja modificado, com o objetivo de ampliar sua área de atuação sem perdas de desempenho ou de qualidade dos resultados. 

\section{Plano de Atividades} 

\begin{enumerate}
\item{\textbf {Disciplinas:}} para minha formação e capacitação no âmbito do projeto, pretendo cursar ao longo dos dois primeiros anos do curso as seguintes disciplinas:

\begin{itemize}
\item Projeto e Análise de Algoritmos 
\item Teoria dos Grafos
\item Otimização Combinatória 
\item Recuperação de Informação
\item Mineração de Dados
\item Aprendizado de Máquina
\item Sistemas de Recomendação
\item Robótica Móvel
\item Programação Paralela 
\item Bancos de Dados Distribuídos 
\end{itemize}

\begin{comment}
-- Outras disciplinas interessantes
Seminários Avançados em Recuperação de Informação e Computação Social
Bancos de Dados Distribuídos 
Otimização em Redes 
Estágio em Docência I
Estágio em Docência II
Computação Ubíqua
Computação Natural 
Compiladores 
Programação Linear 
Fundamentos Teóricos da Computação
\end{comment}

\item{\textbf {Exames de Qualificação:}} para o processo de qualificação no doutorado, pretendo realizar os exames de Teoria e de Sistemas de Computação no segundo e terceiro semestre, respectivamente.

\item{\textbf {Período no Exterior:}} tendo em mente que grupos de pesquisa no exterior podem contribuir consideravelmente na qualidade do projeto de doutorado, pretendo trabalhar de dois a três semestres junto a um grupo de pesquisa internacional especializado na minha área de atuação. Planejo o início desta etapa para o terceiro ano de doutorado, após ter sido aprovado nos exames de qualificação.

\item{\textbf {Projeto:}} pesquisa, implementação e testes do projeto, tendo em mente os objetivos descritos na seção anterior, durante o segundo e terceiro ano de doutorado.

\item{\textbf {Elaboração e Defesa da Tese:}} redação da tese durante o quarto ano de curso.

\end {enumerate}

{\footnotesize \center
\ \\ 
\begin{tabular}{| c | c | c | c | c | c | c | c | c |}
\hline
\multicolumn{9}{| c |}{\textbf{\normalsize Desenvolvimento do Trabalho}} \\[2pt] \hline
Etapa & 2014/2 & 2015/1 & 2015/2 & 2016/1 & 2016/2 & 2017/1 & 2017/2 & 2018/1 \\ \hline
         1 &    X   &    X   &    X   &    X   &        &        &        &        \\ \hline
         2 &        &    X   &    X   &        &        &        &        &        \\ \hline
         3 &        &        &        &        &    X   &    X   &    X   &        \\ \hline
         4 &        &    X   &    X   &    X   &    X   &    X   &    X   &        \\ \hline
         5 &        &        &        &        &        &        &    X   &    X   \\ \hline
\end{tabular}
}

\newpage  
\bibliographystyle{plain}%amsalpha
\bibliography{bibliografia}
\newpage

\end{document}


