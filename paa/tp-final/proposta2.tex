\documentclass[a4paper,12pt,titlepage]{article}
%\usepackage[T1]{fontenc}

\usepackage{titlesec}
\usepackage{titling}
\usepackage[portuguese]{babel}
\usepackage[utf8x]{inputenc}
\usepackage{indentfirst}
\usepackage{graphicx}
%\usepackage{times}
\usepackage{ucs}
\usepackage{float}    
\usepackage{fancyvrb}   
\usepackage{verbatim}
\usepackage{listings}
\usepackage{hyperref}
\usepackage{epigraph}
\usepackage{listings}
\usepackage{tabularx}
\usepackage{lipsum}

\hypersetup{
    colorlinks=true,       
    linkcolor=black,          
    citecolor=black,   
    filecolor=black,  
    urlcolor=black  
}

\hyphenation {di-re-cio-na-men-to} 

%\renewcommand*{\familydefault}{\ttdefault}
\lstset{columns=fullflexible,basicstyle=\ttfamily}

\title{\large
Universidade Federal de Minas Gerais \\ \
Instituto de Ciências Exatas \\ \ 
Departamento de Ciência da Computação \\ \
\\[1cm]
Projeto de Final de Curso\\ \
Projeto e Análise de Algoritmos\\ \
\\[1cm]
\textbf{\Large Proposta de Trabalho }
\\[1cm]
}

\author{\large Alberto Hideki Ueda \\[0.5cm] 
	Orientador: Berthier Ribeiro Neto 
\\[3cm] }

\date{\textsc{Belo Horizonte\\ \
7 de outubro de 2014}}

\begin{document}
\maketitle

\pagebreak


% Qual o problema?
O primeiro coletor \textit{Web} conhecido foi criado em 1993 por Matthew Gray, então graduando do MIT, e chamava-se WWWW (\textit{World Wide Web Wanderer). Comprovando a forte relação com a história dos sistemas de busca na \textit{Web}, no mesmo ano foi lançada também a primeira máquina de busca conhecida, ALIWEB, criada por Martijn Koster. Nesta época, um número razoável de servidores que deveriam ser analisados para se obter uma boa cobertura da rede girava em torno de apenas alguns milhares. Desde então, o número de \textit{hosts} tem aumentado em alta velocidade - chegando a praticamente dobrar a cada ano, de 1993 a 1996 -, tornando as máquinas de busca ainda mais necessárias e, consequentemente, também os coletores de dados.


% Referências

\newpage  
\bibliographystyle{plain}%amsalpha
\bibliography{bibliografia}
\newpage

\end{document}


